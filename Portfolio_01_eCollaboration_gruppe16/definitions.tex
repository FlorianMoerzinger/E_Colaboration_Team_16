
\usepackage[utf8]{inputenc}
\usepackage{graphicx}
\usepackage[tmargin=1in,bmargin=1in,lmargin=1.25in,rmargin=1.25in]{geometry}
\usepackage{titlesec}
\usepackage{xcolor}
\usepackage[overload]{textcase}

\usepackage{exscale,relsize}
\usepackage{fancyhdr}
\usepackage[small]{caption}
\usepackage{float}

\usepackage{amsmath}
\usepackage{wallpaper}
\usepackage{pst-all}
\usepackage{amssymb}
\usepackage{rotating}
\usepackage{mathrsfs}
\usepackage{pgfplots}
\usepackage{makeidx}
%\usepackage[pdftex,bookmarks=true,bookmarksnumbered=true]{hyperref}

% equations



\usepackage{amsxtra}
\usepackage{xfrac}




\newcommand{\babel}[2]{\ifthenelse{\boolean{english}}{#1}{#2}}

% color set
\definecolorseries{foo}{rgb}{last}[rgb]{1.0,0.0,0.0}[rgb]{0.0,0.0,1.0}
\resetcolorseries[16]{foo}

% auxillary symbols
\renewcommand{\tilde}{\symbol{126}}
\newcommand{\define}{\stackrel{!}{=}}
\renewcommand{\equiv}{\,\widehat{=}\,}
\newcommand{\subsubsubsection}{\textbf}
\newcommand{\re}{\mathrm{Re}}
\newcommand{\pr}{\mathrm{Pr}}
\newcommand{\st}{\mathrm{St}}
\newcommand{\fr}{\mathrm{Fr}}
\newcommand{\nus}{\mathrm{Nu}}
\newcommand{\gr}{\mathrm{Gr}}
\newcommand{\ra}{\mathrm{Ra}}
\newcommand{\mif}{\quad\mathrm{\babel{if}{falls}}\quad}
\newcommand{\with}{\quad\mathrm{\babel{with}{mit}}\quad}
\newcommand{\for}{\quad\mathrm{\babel{for}{f"ur}}\quad}
%\renewcommand{\not}{\not}
\newcommand{\im}{i}
\newcommand{\ariwam}{ARiWaM}
\newcommand{\matlab}{MATLAB}

% format specifications
\renewcommand{\emph}{\textbf}
\newcommand{\file}{\textit}
\newcommand{\cmd}{\texttt}
\newcommand{\ten}{\boldsymbol}
%\newcommand{\unit}{\mathrm}
\newcommand{\lemma}{\textit}
\newcommand{\deutsch}[1]{german: \textit{#1}}
\renewcommand{\index}{\emph}

% Command path to graphic files
\newcommand{\gpath}{./grafics}
\newcommand{\bsppath}{../uebungen/beispiele}

% mathematical operators
\newcommand{\grad}{\,\mathrm{grad}\,}
\renewcommand{\div}{\,\mathrm{div}\,}
\newcommand{\rot}{\,\mathrm{rot}\,}
\newcommand{\lap}{\Delta}
\newcommand{\laplace}[1]{\mathscr{L}\left\{#1\right\}}
\newcommand{\trans}{^T}
\newcommand{\norm}{\psarc[linewidth=0.5pt](0,0){0.4}{0}{90}\psdot[dotsize=0.1](0.15,0.15)}

%\renewcommand{\labelenumi}{\alph{enumi})}

\setlength{\parindent}{0em}
\setlength{\parskip}{1.5ex plus0.5ex minus0.5ex}
\setlength{\captionmargin}{3em}

% counters
\newcounter{example}
\ifthenelse{\boolean{english}}{\newcommand{\exampletext}{example }}{\newcommand{\exampletext}{Beispiel }}
\newcommand{\example}[1]{\underline{\exampletext \arabic{chapter}.\arabic{example}:} #1 \addtocounter{example}{1}}
\newcounter{exercise}
\ifthenelse{\boolean{english}}{\newcommand{\exercisetext}{exercise }}{\newcommand{\exercisetext}{Aufgabe }}
\newcommand{\exercise}[1]{\underline{\exercisetext \arabic{chapter}.\arabic{exercise}:} #1 \stepcounter{exercise}}
\newcommand{\cchapter}[1]{\chapter{#1} \setcounter{example}{1} \setcounter{exercise}{1}}
\ifthenelse{\boolean{english}}{\newcommand{\solution}{\textit{solution:} }}{\newcommand{\solution}{\textit{L\"osung:} }}
%\newcounter{beispiel}
%\setcounter{beispiel}{1}		% Nummer des ersten Beispiels
\newboolean{student}
\newcounter{enumcount}
\newcommand{\resume}[1]{\begin{#1} \setcounter{enumi}{\value{enumcount}}}
\newcommand{\pause}[1]{\setcounter{enumcount}{\value{enumi}} \end{#1}}

%Matlab code schreiben

\usepackage{listings}
\usepackage{color}
\usepackage{microtype}

\definecolor{green}{RGB}{28,172,0} % color values Red, Green, Blue
\definecolor{mauve}{RGB}{170,55,241}
\definecolor{lightgray}{RGB}{230,230,230}
\definecolor{gray}{RGB}{160,160,160}
\definecolor{dkgray}{RGB}{105,105,105}

\lstset{ %
  backgroundcolor=\color{white},   % choose the background color; you must add \usepackage{color} or \usepackage{xcolor}
  basicstyle=\footnotesize,        % the size of the fonts that are used for the code
  breakatwhitespace=false,         % sets if automatic breaks should only happen at whitespace
  breaklines=true,                 % sets automatic line breaking
  captionpos=t,                    % sets the caption-position to bottom
  commentstyle=\color{green},    % comment style
  deletekeywords={...},            % if you want to delete keywords from the given language
  escapeinside={\%*}{*)},          % if you want to add LaTeX within your code
  extendedchars=true,              % lets you use non-ASCII characters; for 8-bits encodings only, does not work with UTF-8
  frame=single,                    % adds a frame around the code
  keepspaces=true,                 % keeps spaces in text, useful for keeping indentation of code (possibly needs columns=flexible)
  keywordstyle=\color{blue},       % keyword style
  xleftmargin=11pt,          % linker Abstand vom Rand (framesep+framrule)
  xrightmargin=5pt,
  language=Octave,                 % the language of the code
  morekeywords={*,...},            % if you want to add more keywords to the set
  numbers=left,                    % where to put the line-numbers; possible values are (none, left, right)
  numbersep=6pt,                   % how far the line-numbers are from the code
  numberstyle=\tiny\color{dkgray}, % the style that is used for the line-numbers
  rulecolor=\color{gray},         % if not set, the frame-color may be changed on line-breaks within not-black text (e.g. comments (green here))
  showspaces=false,                % show spaces everywhere adding particular underscores; it overrides 'showstringspaces'
  showstringspaces=false,          % underline spaces within strings only
  showtabs=false,                  % show tabs within strings adding particular underscores
  stepnumber=2,                    % the step between two line-numbers. If it's 1, each line will be numbered
  stringstyle=\color{mauve},     % string literal style
  tabsize=2,                       % sets default tabsize to 2 spaces
  title=\lstname                   % show the filename of files included with \lstinputlisting; also try caption instead of title
}

%% Package für Code-listings (Syntayhighlight)
\usepackage{listings}
%% erweiterte Bildunterschriften
\usepackage{caption}
\usepackage{subcaption}

%% Counter für die römische Nummerierung
\newcounter{romancount}

% Einheit nicht kursiv schreieben \SI{5}{kg} \si{kg}
\usepackage{siunitx}
%% Kapitelnummer in Formel-, Abbildungs- und Tabellennummerierung integrieren
\numberwithin{equation}{section}
\numberwithin{figure}{section}
\numberwithin{table}{section}


%Abstand zwischen Überschrift und Tabelle
\setlength\belowcaptionskip{10pt}
\usepackage{bm}
\renewcommand{\arraystretch}{1.5}	
%Zeilenabstand 1.5-fache
\usepackage{setspace}
\onehalfspacing


\usepackage{multirow}
\usepackage{bigdelim}
\usepackage{pdfpages} 
